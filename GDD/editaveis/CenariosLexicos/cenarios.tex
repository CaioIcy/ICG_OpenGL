\section{Cenários}\index{Cenários}

Nessa seção, serão descritos os \hypertarget{cenarios}{cenários} do jogo Alke Bike.

\subsection{Cenário 1}\index{Cenário 1}

\textbf{Título:} Dinâmica do Jogo no modo \hyperlink{endurance}{\textit{Endurance}}.

\textbf{Objetivo:} Descrever as regras do Alke Bike no modo \textit{Endurance}.

\textbf{Contexto:} Início de Jogo.

\textbf{Atores:} Jogador.

\textbf{Recursos:} Personagem, \hyperlink{blocoDeComando}{Bloco de Comando}.

\textbf{Episódios:} Jogador inicia o jogo. Jogador seleciona o nível de dificuldade. Aparece um tempo regressivo de preparação para o Jogador. Jogador seleciona o Bloco de Comando para pedalar caso o bloco esteja na \hyperlink{areaDeComando}{Área de Comando}. Caso selecione o Bloco correto, o bloco muda de cor temporariamente e o jogador seleciona o próximo bloco quando este estiver na Área de Comando sendo que os Blocos vão chegando na Área de Comando mais rápido de acordo com o tempo, caso contrário, o jogo acaba e o \hyperlink{score}{\textit{score}} final dele é mostrado. Caso o score seja o maior do Jogador o \textit{high score} é registrado.

\subsection{Cenário 2}\index{Cenário 2}

\textbf{Título:} Dinâmica do Jogo no modo \hyperlink{dexterity}{\textit{Dexterity}}.

\textbf{Objetivo:} Descrever as regras do Alke Bike no modo \textit{Dexterity}.

\textbf{Contexto:} Início de Jogo.

\textbf{Atores:} Jogador.

\textbf{Recursos:} Personagem, Bloco de Comando, Quantidade de Blocos de Comando.

\textbf{Episódios:} Jogador inicia o jogo. Jogador seleciona o nível de dificuldade. Aparece um tempo regressivo de preparação para o Jogador. Jogador seleciona o Bloco de Comando para pedalar caso o Bloco esteja na Área de Comando. Caso o jogador acerte o bloco, o bloco muda de cor temporariamente e o próximo Bloco entra na Área de Comando. Caso o jogador erre o comando ou acabe a quantidade de Blocos a serem pressionados, o jogo acaba e o \textit{score} final dele é mostrado. Caso o score seja o maior do Jogador o \textit{high score} é registrado.

\subsection{Cenário 3}\index{Cenário 3}

\textbf{Título:} Iniciação de Jogo.

\textbf{Objetivo:} Descrever como o jogo é iniciado.

\textbf{Contexto:} Aplicativo começa a ser executado.

\textbf{Atores:} Jogador.

\textbf{Recursos:} Menu, Personagem e Blocos de Comando.

\textbf{Episódios:} Jogador inicia o aplicativo. Jogador seleciona o botão Jogar. Jogador escolhe o modo de jogo. Jogador escolhe nível de dificuldade. Jogo carrega. Jogo apresenta tempo regressivo de preparaçao. Jogador inicia o jogo.

\subsection{Cenário 4}\index{Cenário 4}

\textbf{Título:} Visualização dos \textit{high scores}.

\textbf{Objetivo:} Descrever como o jogador visualiza os \textit{high scores} obtidas por ele.

\textbf{Contexto:} Aplicativo começa a ser executado.

\textbf{Atores:} Jogador.

\textbf{Recursos:} Menu, Lista de \textit{high scores}.

\textbf{Episódios:} Jogador inicia o aplicativo. Jogador seleciona o botão \textit{High Scores}. Jogo apresenta os \textit{High Scores} separados por modo de jogo.

\subsection{Cenário 5}\index{Cenário 5}

\textbf{Título:} Visualização dos créditos do jogo.

\textbf{Objetivo:} Descrever como o jogador assiste aos créditos finais do jogo.

\textbf{Contexto:} Aplicativo começa a ser executado ou após a derrota do jogador.

\textbf{Atores:} Jogador

\textbf{Recursos:} Menu, Créditos finais.

\textbf{Episódios:} Jogador inicia o aplicativo. Jogador escolhe o botão Créditos. Jogo apresenta os Créditos.

\subsection{Cenário 6}\index{Cenário 6}

\textbf{Título:} Alteração de volume do jogo.

\textbf{Objetivo:} Descrever como o jogador altera o volume do jogo.

\textbf{Contexto:} Aplicativo começa a ser executado.

\textbf{Atores:} Jogador

\textbf{Recursos:} Menu, Menu de Opções.

\textbf{Episódios:} Jogador inicia o aplicativo. Jogador escolhe o botão Opções. Jogador seleciona o volume do jogo. Jogador altera o volume do jogo.

\subsection{Cenário 7}\index{Cenário 7}

\textbf{Título:} Alteração da Resolução da Tela.

\textbf{Objetivo:} Descrever como o jogador altera a resolução da tela do jogo.

\textbf{Contexto:} Aplicativo começa a ser executado.

\textbf{Atores:} Jogador.

\textbf{Recursos:} Menu, Menu de Opções.

\textbf{Episódios:} Jogador inicia o aplicativo. Jogador escolhe o botão Opções. Jogador seleciona resolução da tela. Jogador escolhe entre fullscreen e {resolução a ser definida}.



