\section{Introdução} \index{Introdução}
\subsection{Objetivo}\index{Objetivo}

Esse Plano de Iteração descreve os planos do desenvolvimento do jogo Alke Bike.

\subsection{Escopo}\index{Escopo}

Esse Plano de Iteração especifica o planejamento do tempo e recursos nas fases de iniciação, elaboração, construção e transição do desenvolvimento do jogo Alke Bike pela empresa Alke Games.

\subsection{Visão Geral}\index{Visão Geral}

O plano de Iteraçao explicita a distribuiçao dos recursos nas tarefas ao longo do tempo util do projeto. Especificando as quatro fases do projeto (Concepçao, Elaboraçao, Construçao e Transiçao) e situando os envolvidos no andamento do projeto.

\section{Plano}\index{Plano}

\subsection{Recursos}\index{Recursos}

\begin{table}[h]
\begin{tabular}{llll}
\hline
\textbf{Nome da tarefa}                 & \textbf{Início} & \textbf{Conclusão} & \textbf{Nomes dos Recursos}                                                                             \\ \hline
\textbf{Marcos}                         &                 &                    &                                                                                                         \\
Início                                  & 09/03/2015      & 09/03/2015         &                                                                                                         \\
Fase de Concepção                       & 16/03/2015      & 30/03/2015         &                                                                                                         \\
Fase de Elaboração                      & 06/04/2015      & 06/04/2015         &                                                                                                         \\
Fase de Construção                      & 09/03/2015      & 11/06/2015         &                                                                                                         \\
Fase de Transição                       & 12/06/2015      & 19/06/2015         &                                                                                                         \\
                                        &                 &                    &                                                                                                         \\
\textbf{Modelagem do Negócio}           &                 &                    &                                                                                                         \\
Capturar vocabulário comum              & 16/03/2015      & 30/03/2015         & João Paulo                                                                                              \\
Definir cenários                        & 16/03/2015      & 23/03/2015         & João Paulo                                                                                              \\
Específicar cenários e léxicos          & 24/03/2015      & 30/03/2015         & João Paulo                                                                                              \\
                                        &                 &                    &                                                                                                         \\
\textbf{Requisitos}                     &                 &                    &                                                                                                         \\
Desenvolver o GDD                       & 16/03/2015      & 30/03/2015         & \begin{tabular}[c]{@{}l@{}}Caio Nardelli, João Paulo,  \\ Matheus Godinho, Simião Carvalho\end{tabular} \\
Priorizar cenários                      & 24/03/2015      & 30/03/2015         & Simião Carvalho                                                                                         \\
Definir restrições do sistema           & 16/03/2015      & 30/03/2015         & Matheus Godinho                                                                                         \\
                                        &                 &                    &                                                                                                         \\
\textbf{Gerenciamento de Configuração}  &                 &                    &                                                                                                         \\
Estabelecer práticas de GC              & 16/03/2015      & 30/03/2015         & Caio Nardelli                                                                                           \\
Estabelecer ambiente de GC              & 16/03/2015      & 30/03/2015         & Caio Nardelli                                                                                           \\
                                        &                 &                    &                                                                                                         \\
\textbf{Modelagem do Negócio}           &                 &                    &                                                                                                         \\
Elaborar modelo de domínio              & 06/04/2015      & 10/04/2015         & Matheus Godinho                                                                                         \\
Desenvolver diagrama de sequência       & 06/04/2015      & 13/04/2015         & João Paulo                                                                                              \\
                                        &                 &                    &                                                                                                         \\
\textbf{Análise e Design}               &                 &                    &                                                                                                         \\
Análise priorizada dos cenários         & 06/04/2015      & 13/04/2015         & \begin{tabular}[c]{@{}l@{}}Caio Nardelli, João Paulo, \\ Matheus Godinho, Simião Carvalho\end{tabular}  \\
Análise de arquitetura                  & 06/04/2015      & 13/04/2015         & Caio Nardelli, Simião Carvalho                                                                          \\
                                        &                 &                    &                                                                                                         \\
\textbf{Implementação de arquitetura}   &                 &                    &                                                                                                         \\
Estruturar implementação de arquitetura & 06/04/2015      & 13/04/2015         & Caio Nardelli, Simião Carvalho                                                                          \\
Testar implementação de arquitetura     & 06/04/2015      & 13/04/2015         & Caio Nardelli, Simião Carvalho                                                                          \\
                                        &                 &                    &                                                                                                         \\
\textbf{Gerenciamento}                  &                 &                    &                                                                                                         \\
Atualizar plano de iteração             & 06/04/2015      & 13/04/2015         & João Paulo, Matheus Godinho                                                                             \\
Avaliar iteração                        & 13/04/2015      & 13/04/2015         & \begin{tabular}[c]{@{}l@{}}Caio Nardelli, João Paulo, \\ Matheus Godinho, Simião Carvalho\end{tabular}  \\
                                        &                 &                    &                                                                                                         \\ \hline
\end{tabular}
\end{table}

\subsection{Cenários}\index{Cenários}

Durante a Iteração de Concepçao, todos os cenarios serão identificados. Os objetivos, contexto, atores, recursos e episodios serão determinados e documentados nas Especificações dos Cenarios e Lexicos. A implementação dos cenarios será iniciado na próxima iteração.

\subsection{Critérios de Avaliação}\index{Critérios de Avaliação}

O principal objetivo da Iteração de Concepçao é definir o sistema para o nível de detalhes requerido para uma boa compreensao do projeto a partir de uma perspectiva de viabilidade de desenvolvimento. Quando a iteração for concluída, uma revisão da Concepçao com foco na qualidade, tempo e escopo chegará a uma decisão Aprovado / Não Aprovado para o projeto.
